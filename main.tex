\documentclass{article}
\usepackage[a4paper, top=20mm, bottom=20mm]{geometry}
\usepackage{amssymb}
\usepackage{amssymb}
\usepackage{amsmath}
\usepackage{mathtools}
\usepackage{graphicx}

\begin{document}

  \section{Punti e Retta}

  \subsection*{Distanza punto-punto}
  \begin{equation}
    d = \sqrt{(x_1 - x_2)^2 + (y_1 - y_2)^2}
  \end{equation}

  \subsection*{Punto medio}
  \begin{equation}
    P_M = \left(\frac{x_1 + x_2}{2}; \frac{y_1 + y_2}{2}\right)
  \end{equation}

  \subsection*{Retta - forma implicita}
  \begin{equation}
    ax+by+c = 0
  \end{equation}

  \subsection*{Retta - forma esplicita}
  \begin{equation}
    y = mx + q
  \end{equation} 
  Descrive tutte le rette tranne quelle parallele all'asse $y$. Il parametro $m$ è il coefficiente angolare e il parametro $q$ è il termine noto.

  \subsection*{Distanza punto-retta}
  Si consideri il punto $P_1=(x_1;y_1)$ e la retta $ax+by+c=0$
  \begin{equation}
    d = \frac{|ax_1 +by_1 + c|}{\sqrt{a^2+b^2}}
  \end{equation}

  \subsection*{Retta passante per 2 punti}
  \begin{equation}
    \frac{y-y_1}{y_2-y_1} = \frac{x-x_1}{x_2-x_1}
    \qquad\text{oppure}\qquad
    y = \frac{x_2-x_1}{y_2-y_1}(x-x_1) + y_1
  \end{equation}
  Questa equazione vale solo nei casi in cui $y_2-y_1 \ne 0$ e $x_2-x_1 \ne 0$. In questi 2 casi particolari abbiamo semplicemente 2 rette particolari: la prima parallela all'asse $x$ mentre la seconda parallela all'asse $y$.

  \subsection*{Fascio di rette per un punto}
  \begin{equation}
    y-y_1 = m(x-x_1) \qquad\cup\qquad x = x_1
  \end{equation}
  Al variare di $m \in \mathbb{R}$ descrive tutte le rette passanti per il punto eccetto quella parallela all'asse $y$.

  \subsection*{Fasio di rette generato da 2 rette}
  Date le rette $a_1x+b_1y+c_1 = 0$ e $a_2x+b_2y+c_2 = 0$
  \begin{equation}
    a_1x+b_1y+c_1 + k (a_2x+b_2y+c_2) = 0 \qquad\cup\qquad a_2x+b_2y+c_2 = 0
  \end{equation}
  Al variare di $k \in \mathbb{R}$ descrive tutte le rette del fascio eccetto la retta $a_2x+b_2y+c_2 = 0$.

  \newpage
  \section{Circonferenza}
  \subsection*{Circonferenza - equazione con centro e raggio}
  \begin{equation}
    (x-x_c)^2+(y-y_c)^2 = r^2
  \end{equation}

  \subsection*{Circonferenza - forma implicita}
  \begin{equation}
    x^2+y^2+ax+by+c = 0
  \end{equation}
  I parametri $a, b, c$ sono legati alle coordinate del centro e al raggio in questo modo: 
  \begin{align*}
    x_c &= -\frac{a}{2}\\
    y_c &= -\frac{b}{2}\\
    r^2 &= \frac{a^2}{2} + \frac{b^2}{2} - c^2 \\
  \end{align*}
  Affinché l'equazione descriva una circonferenza e quindi $\frac{a^2}{2} + \frac{b^2}{2} - c^2 > 0$.

  \subsection*{Asse radiale fra 2 circonferenze}
  Date le circonferenze $x^2+y^2+a_1x+b_1y+c_1 = 0$ e $x^2+y^2+a_2x+b_2y+c_2 = 0$ 
  \begin{equation}
    (a_1-a_2)x+(b_1-b_2)y+c_1-c_2 = 0
  \end{equation}
  L'asse radiale esiste solo se le 2 circonferenze non sono concentriche.

  \subsection*{Fascio di circonferenze generato da 2 circonferenze}
  \begin{equation}
    x^2+y^2+a_1x+b_1y+c_1 +k (x^2+y^2+a_2x+b_2y+c_2) = 0
  \end{equation}

  Al variare di $k \in \mathbb{R}$ descrive tutte le circonferenze del fascio eccetto quella che viene moltiplicata per il parametro $k$ $x^2+y^2+a_2x+b_2y+c_2$.

  L'asse radiale è una particolare circonferenza del fascio e si ottiene per $k=-1$.

\end{document}
