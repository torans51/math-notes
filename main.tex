\documentclass{article}
\usepackage[a4paper, top=20mm, bottom=20mm]{geometry}
\usepackage{amssymb}
\usepackage{amssymb}
\usepackage{amsmath}
\usepackage{mathtools}
\usepackage{graphicx}

\let\svthefootnote\thefootnote
\newcommand\blankfootnote[1]{%
  \let\thefootnote\relax\footnotetext{#1}%
  \let\thefootnote\svthefootnote%
}

\begin{document}
  \section{Punti e Retta}

  \subsection*{Distanza punto-punto}
  \begin{equation}
    d = \sqrt{(x_1 - x_2)^2 + (y_1 - y_2)^2}
  \end{equation}

  \subsection*{Punto medio}
  \begin{equation}
    P_M = \left(\frac{x_1 + x_2}{2}; \frac{y_1 + y_2}{2}\right)
  \end{equation}

  \subsection*{Retta - forma implicita}
  \begin{equation}
    ax+by+c = 0
  \end{equation}

  \subsection*{Retta - forma esplicita}
  \begin{equation}
    y = mx + q
  \end{equation} 
  Descrive tutte le rette tranne quelle parallele all'asse $y$. Il parametro $m$ è il coefficiente angolare e il parametro $q$ è il termine noto.

  \subsection*{Distanza punto-retta}
  Si consideri il punto $P_1=(x_1;y_1)$ e la retta $ax+by+c=0$
  \begin{equation}
    d = \frac{|ax_1 +by_1 + c|}{\sqrt{a^2+b^2}}
  \end{equation}

  \subsection*{Retta passante per 2 punti}
  \begin{equation}
    \frac{y-y_1}{y_2-y_1} = \frac{x-x_1}{x_2-x_1}
    \qquad\text{oppure}\qquad
    y = \frac{x_2-x_1}{y_2-y_1}(x-x_1) + y_1
  \end{equation}
  Questa equazione vale solo nei casi in cui $y_2-y_1 \ne 0$ e $x_2-x_1 \ne 0$. In questi 2 casi particolari abbiamo semplicemente 2 rette particolari: la prima parallela all'asse $x$ mentre la seconda parallela all'asse $y$.

  \subsection*{Fascio di rette per un punto}
  \begin{equation}
    y-y_1 = m(x-x_1) \qquad\cup\qquad x = x_1
  \end{equation}
  Al variare di $m \in \mathbb{R}$ descrive tutte le rette passanti per il punto eccetto quella parallela all'asse $y$.

  \subsection*{Fasio di rette generato da 2 rette}
  Date le rette $a_1x+b_1y+c_1 = 0$ e $a_2x+b_2y+c_2 = 0$
  \begin{equation}
    a_1x+b_1y+c_1 + k (a_2x+b_2y+c_2) = 0 \qquad\cup\qquad a_2x+b_2y+c_2 = 0
  \end{equation}
  Al variare di $k \in \mathbb{R}$ descrive tutte le rette del fascio eccetto la retta $a_2x+b_2y+c_2 = 0$.

  \newpage
  \section{Circonferenza}

  \subsection*{Circonferenza - equazione con centro e raggio}
  \begin{equation}
    (x-x_c)^2+(y-y_c)^2 = r^2
  \end{equation}

  \subsection*{Circonferenza - forma implicita}
  \begin{equation}
    x^2+y^2+ax+by+c = 0
  \end{equation}
  I parametri $a, b, c$ sono legati alle coordinate del centro e al raggio in questo modo: 
  \begin{align*}
    x_c &= -\frac{a}{2}\\
    y_c &= -\frac{b}{2}\\
    r^2 &= \frac{a^2}{4} + \frac{b^2}{4} - c \\
  \end{align*}
  Affinché l'equazione descriva una circonferenza $\frac{a^2}{4} + \frac{b^2}{4} - c \ge 0$.

  \subsection*{Asse radiale fra 2 circonferenze}
  Date le circonferenze $x^2+y^2+a_1x+b_1y+c_1 = 0$ e $x^2+y^2+a_2x+b_2y+c_2 = 0$ 
  \begin{equation}
    (a_1-a_2)x+(b_1-b_2)y+c_1-c_2 = 0
  \end{equation}
  L'asse radiale esiste solo se le 2 circonferenze non sono concentriche.

  \subsection*{Fascio di circonferenze generato da 2 circonferenze}
  \begin{equation}
    x^2+y^2+a_1x+b_1y+c_1 +k (x^2+y^2+a_2x+b_2y+c_2) = 0
  \end{equation}

  Al variare di $k \in \mathbb{R}$ descrive tutte le circonferenze del fascio eccetto quella che viene moltiplicata per il parametro $k$: $x^2+y^2+a_2x+b_2y+c_2 = 0$.

  L'asse radiale è una particolare circonferenza del fascio e si ottiene per $k=-1$.

  \newpage
  \section{Parabola}

  \subsection*{Parabola - equazione e caratteristiche principali}
  \begin{minipage}[t]{.45\textwidth}
    Parabola con asse di simmetria parallelo all'asse $y$.
    \begin{align}
      & \text{\textit{equazione:}} \qquad
      y = ax^2 + bx + c \\
      & \text{\textit{asse di simmetria:}} \qquad
      x = -\frac{b}{2a} \\
      & \text{\textit{vertice:}} \qquad
      V = \left(-\frac{b}{2a}; -\frac{\Delta}{4a}\right) \\
      & \text{\textit{fuoco:}} \qquad
      F = \left(-\frac{b}{2a}; \frac{1}{4a} -\frac{\Delta}{4a}\right) \\
      & \text{\textit{direttrice:}} \qquad
      y = -\frac{1}{4a} -\frac{\Delta}{4a}
    \end{align}
  \end{minipage}
  \hfill
  \begin{minipage}[t]{.45\textwidth}
    Parabola con asse di simmetria parallelo all'asse $x$.
    \begin{align}
      & \text{\textit{equazione:}} \qquad
      x = ay^2 + bx + c \\
      & \text{\textit{asse di simmetria:}} \qquad
      y = -\frac{b}{2a} \\
      & \text{\textit{vertice:}} \qquad
      V = \left(-\frac{\Delta}{4a}; -\frac{b}{2a}\right) \\
      & \text{\textit{fuoco:}} \qquad
      F = \left(\frac{1}{4a} -\frac{\Delta}{4a}; -\frac{b}{2a}\right) \\
      & \text{\textit{direttrice:}} \qquad
      x = -\frac{1}{4a} -\frac{\Delta}{4a}
    \end{align}
  \end{minipage}

  \subsection*{Fascio di parabole generato da 2 parabole}
  \begin{equation}
    y -a_1x^2 -b_1x -c_1 +k(y -a_2x^2 -b_2x -c_2) = 0
  \end{equation}

  Al variare di $k \in \mathbb{R}$ descrive tutte le parabole del fascio eccetto quella che viene moltiplicata per il parametro $k$: $y - a_2x^2 -b_2x -c_2 = 0$.

  Scegliendo $k$ in modo tale che i termini in $x^2$ si annullino si ottiene l'asse del fascio.

  Scegliendo $k=-1$ si ottiene un'equazione di secondo grado in $x$ che permette di ricavare le ascisse dei punti base. 
  \begin{itemize}
    \item 2 soluzioni distinte: le parabole si intersecano sempre nei 2 punti base e l'asse del fascio passa per i 2 punti base
    \item 2 soluzioni coincidenti: le parabole sono tangenti nel punto base e l'asse del fascio è la retta tangente a tutte le parabole per il punto base
    \item 1 soluzione: le parabole sono congruenti, si intersecano in un punto e gli assi di simmetria sono tutti paralleli fra loro.
    \item nessuna soluzione: le parabole sono congruenti e non si intersecano fra loro
  \end{itemize}

  \subsection*{Fascio di parabole generato da 2 punti}
  Dati i punti $P_1=(x_1; y_1)$ e $P_2=(x_2; y_2)$ e $y=mx+q$ la retta passante per i punti $P_1$ e $P_2$
  \begin{equation}
    y = mx +q +k(x-x_1)(x-x_2)
  \end{equation}

  \subsection*{Fascio di parabole generato da una retta e un punto sulla retta}
  Data la retta $y=mx+q$ e un punto $P_1=(x_1;y_1)$ appartenente ad essa
  \begin{equation}
    y=mx+q + k(x-x_1)^2
  \end{equation}

  \newpage
  \section{Ellisse}

  \subsection*{Ellisse - equazione e caratteristiche principali}
  \begin{minipage}[t]{.45\textwidth}
    Ellisse con centro l'origine e con i fuochi sull'asse $x$.
    \begin{align}
      & \text{\textit{equazione:}} \qquad
      \frac{x^2}{a^2} + \frac{y^2}{b^2} = 1 \\
      & \text{\textit{semiasse maggiore:}} \quad a \\
      & \text{\textit{semiasse minore:}} \quad b \\
      & \text{\textit{posizione dei fuochi:}} \nonumber \\
      & \quad F_1 = (-c, 0) \quad F_2 = (c; 0) \\
      & \text{\textit{relazione fuochi e semiassi:}} \nonumber \\
      & \quad a^2 = b^2 + c^2 \\
      & \text{\textit{eccentricità:}} \nonumber \\
      & \quad e = \frac{c}{a} = \sqrt{1-\frac{b^2}{a^2}}
    \end{align}
  \end{minipage}
  \hfill
  \begin{minipage}[t]{.45\textwidth}
    Ellisse con centro l'origine e con i fuochi sull'asse $y$.
    \begin{align}
      & \text{\textit{equazione:}} \qquad
      \frac{x^2}{a^2} + \frac{y^2}{b^2} = 1 \\
      & \text{\textit{semiasse maggiore:}} \quad b \\
      & \text{\textit{semiasse minore:}} \quad a \\
      & \text{\textit{posizione dei fuochi:}} \nonumber \\
      & \quad F_1 = (0, -c) \quad F_2 = (0; c) \\
      & \text{\textit{relazione fuochi e semiassi:}} \nonumber \\
      & \quad b^2 = a^2 + c^2 \\
      & \text{\textit{eccentricità:}} \nonumber \\
      & \quad e = \frac{c}{b} = \sqrt{1-\frac{a^2}{b^2}}
    \end{align}
  \end{minipage}

  \subsection*{Ellisse con centro traslato}
  Ellisse con fuochi sull'asse $x$ o asse $y$ e centro traslato
  \begin{equation}
    \frac{(x-x_c)^2}{a^2} + \frac{(y-y_c)^2}{b^2} = 1 \\
  \end{equation}

  \newpage
  \section{Iperbole}

  \begin{minipage}[t]{.45\textwidth}
    Ellisse con centro l'origine e con i fuochi sull'asse $x$.
    \begin{align}
      & \text{\textit{equazione:}} \qquad
      \frac{x^2}{a^2} - \frac{y^2}{b^2} = 1 \\
      & \text{\textit{semiasse trasverso:}} \quad a \\
      & \text{\textit{semiasse non trasverso:}} \quad b \\
      & \text{\textit{posizione dei fuochi:}} \nonumber \\
      & \quad F_1 = (-c, 0) \quad F_2 = (c; 0) \\
      & \text{\textit{relazione fuochi e semiassi:}} \nonumber \\
      & \quad c^2 = a^2 + b^2 \\
      & \text{\textit{eccentricità:}} \nonumber \\
      & \quad e = \frac{c}{a} = \sqrt{1+\frac{b^2}{a^2}} \\
      & \text{\textit{asintoti:}} \nonumber \\
      & \quad y=\pm \frac{b}{a}x
    \end{align}
  \end{minipage}
  \hfill
  \begin{minipage}[t]{.45\textwidth}
    Ellisse con centro l'origine e con i fuochi sull'asse $y$.
    \begin{align}
      & \text{\textit{equazione:}} \qquad
      \frac{x^2}{a^2} - \frac{y^2}{b^2} = -1 \\
      & \text{\textit{semiasse trasverso:}} \quad b \\
      & \text{\textit{semiasse non trasverso:}} \quad a \\
      & \text{\textit{posizione dei fuochi:}} \nonumber \\
      & \quad F_1 = (0, -c) \quad F_2 = (0; c) \\
      & \text{\textit{relazione fuochi e semiassi:}} \nonumber \\
      & \quad c^2 = b^2 + a^2 \\
      & \text{\textit{eccentricità:}} \nonumber \\
      & \quad e = \frac{c}{b} = \sqrt{1+\frac{a^2}{b^2}} \\
      & \text{\textit{asintoti:}} \nonumber \\
      & \quad y=\pm \frac{b}{a}x
    \end{align}
  \end{minipage}

  \subsection*{Iperbole con centro traslato}
  Iperbole con fuochi sull'asse $x$ ($+$) o asse $y$ ($-$) e centro traslato
  \begin{equation}
    \frac{(x-x_c)^2}{a^2} - \frac{(y-y_c)^2}{b^2} = \pm 1 \\
  \end{equation}

  \subsection*{Iperbole equilatera con centro traslato}
  Iperbole equilatera con fuochi sull'asse $x$ ($+$) o asse $y$ ($-$) e centro traslato
  \begin{equation}
    \frac{(x-x_c)^2}{a^2} - \frac{(y-y_c)^2}{a^2} = \pm 1 \\
  \end{equation}

  \subsection*{Iperbole riferita agli assi}
  L'iperbole riferita agli assi con centro l'origine è un'iperbole equilatera dove i fuochi sono sulla bisettrice del primo e del terzo quadrante ($+$) oppure sulla bisettrice del secondo e del quarto quadrante ($-$)
  \begin{equation}
    xy = \pm k \qquad \text{\textit{k numero}}
  \end{equation}
  Se il centro fosse traslato l'equazione diventa
  \begin{equation}
    (x-x_c)(y-y_c) = \pm k \qquad \text{\textit{k numero}}
  \end{equation}

  \newpage
  \section{Funzioni e formule goniometriche}
  \blankfootnote{Per le formule con denominatori dipendenti da funzioni trigonometriche o con la tangente bisogna imporre tutte le condizioni di esistenza sugli angoli.}

  \subsection*{Relazioni fondamentali e angoli principali}
  \begin{minipage}[t][][b]{.45\textwidth}
    \begin{equation}
      \sin^2 \alpha + \cos^2 \alpha = 1 \quad \forall \alpha \in \mathbb{R}
    \end{equation}
    \begin{equation}
      \tan\alpha = \frac{\sin\alpha}{\cos\alpha}
    \end{equation}
    \begin{equation}
      \sin\alpha = - \sin(-\alpha)
    \end{equation}
    \begin{equation}
      \cos\alpha = \cos(-\alpha)
    \end{equation}
  \end{minipage}
  \begin{minipage}[t][][b]{.45\textwidth}
    \begin{flushright}
      \begin{tabular}{ | c | c | c | c | c | }
        \hline
        deg & rad & $\sin \alpha$ & $\cos\alpha$ & $\tan\alpha$ \\
        \hline
        $0^o$ & $0$ & $0$ & $1$ & $0$ \\
        $30^o$ & $\frac{\pi}{6}$ & $\frac{1}{2}$ & $\frac{\sqrt{3}}{2}$ & $\frac{\sqrt{3}}{3}$ \\
        $45^o$ & $\frac{\pi}{4}$ & $\frac{\sqrt{2}}{2}$ & $\frac{\sqrt{2}}{2}$ & $1$ \\
        $60^o$ & $\frac{\pi}{3}$ & $\frac{\sqrt{3}}{2}$ & $\frac{1}{2}$ & $\sqrt{3}$ \\
        $90^o$ & $\frac{\pi}{2}$ & $1$ & $0$ & $\text{not def}$ \\
        \hline
      \end{tabular}
    \end{flushright}
  \end{minipage}
  % \begin{equation*}
  %   \sin^2 \alpha + \cos^2 \alpha = 1 \quad \forall \alpha \in \mathbb{R}
  %   \qquad
  %   \tan\alpha = \frac{\sin\alpha}{\cos\alpha}
  % \end{equation*}
  % \renewcommand{\arraystretch}{1.5}
  % \begin{center}
  %   \begin{tabular}{ | c | c | c | c | c | }
  %     \hline
  %     deg & rad & $\sin \alpha$ & $\cos\alpha$ & $\tan\alpha$ \\
  %     \hline
  %     $0^o$ & $0$ & $0$ & $1$ & $0$ \\
  %     $30^o$ & $\frac{\pi}{6}$ & $\frac{1}{2}$ & $\frac{\sqrt{3}}{2}$ & $\frac{\sqrt{3}}{3}$ \\
  %     $45^o$ & $\frac{\pi}{4}$ & $\frac{\sqrt{2}}{2}$ & $\frac{\sqrt{2}}{2}$ & $1$ \\
  %     $60^o$ & $\frac{\pi}{3}$ & $\frac{\sqrt{3}}{2}$ & $\frac{1}{2}$ & $\sqrt{3}$ \\
  %     $90^o$ & $\frac{\pi}{2}$ & $1$ & $0$ & $\text{not def}$ \\
  %     \hline
  %   \end{tabular}
  % \end{center}
  \renewcommand{\arraystretch}{1}

  \subsection*{Formule di addizione/sottrazione}
  \begin{equation}
    \sin(\alpha\pm\beta) = \sin\alpha \, \cos\beta \pm \cos\alpha \, \sin\beta
  \end{equation}
  \begin{equation}
    \cos(\alpha\pm\beta) = \cos\alpha \, \cos\beta \mp \sin\alpha \, \sin\beta
  \end{equation}
  \begin{equation}
    \tan(\alpha\pm\beta) = \frac{\tan\alpha \pm \\tan\beta}{1 \mp \tan\alpha \, \tan\beta}
  \end{equation}

  \subsection*{Formule di duplicazione}
  \begin{equation}
    \sin(2\alpha) = 2 \sin\alpha \, \cos\alpha
  \end{equation}
  \begin{equation}
    \cos(2\alpha) = \cos^2 \alpha - \sin^2 \alpha
  \end{equation}
  \begin{equation}
    \tan(2\alpha) = \frac{2\tan\alpha}{1-\tan^2\alpha}
  \end{equation}

  \subsection*{Formule di bisezione}
  \begin{equation}
    \sin\frac{\alpha}{2} = \pm \sqrt{\frac{1-\cos\alpha}{2}}
    \qquad
    \cos\frac{\alpha}{2} = \pm \sqrt{\frac{1+\cos\alpha}{2}}
    \qquad
    \tan\frac{\alpha}{2} = \pm \sqrt{\frac{1-\cos\alpha}{1+\cos\alpha}}
  \end{equation}
  % \begin{equation}
  %   \sin\frac{\alpha}{2} = \pm \sqrt{\frac{1-\cos\alpha}{2}}
  % \end{equation}
  % \begin{equation}
  %   \cos\frac{\alpha}{2} = \pm \sqrt{\frac{1+\cos\alpha}{2}}
  % \end{equation}
  % \begin{equation}
  %   \tan\frac{\alpha}{2} = \pm \sqrt{\frac{1-\cos\alpha}{1+\cos\alpha}}
  % \end{equation}

  \subsection*{Formule di parametriche}
  \begin{equation}
    \sin\alpha = \frac{2\tan\frac{\alpha}{2}}{1+\tan^2 \frac{\alpha}{2}}
    \qquad
    \cos\alpha = \frac{1-\tan^2 \frac{\alpha}{2}}{1+\tan^2 \frac{\alpha}{2}}
    \qquad
    \tan\alpha = \frac{2\tan\frac{\alpha}{2}}{1-\tan^2 \frac{\alpha}{2}}
  \end{equation}
  % \begin{equation}
  %   \sin\alpha = \frac{2\tan\frac{\alpha}{2}}{1+\tan^2 \frac{\alpha}{2}}
  % \end{equation}
  % \begin{equation}
  %   \cos\alpha = \frac{1-\tan^2 \frac{\alpha}{2}}{1+\tan^2 \frac{\alpha}{2}}
  % \end{equation}

  \subsection*{Formule di prostaferesi}
  \begin{equation}
    \sin\alpha \pm \sin\beta = 2 \sin\frac{\alpha \pm \beta}{2} \cos\frac{\alpha \mp \beta}{2}
  \end{equation}
  \begin{equation}
    \cos\alpha + \cos\beta = 2 \cos\frac{\alpha + \beta}{2} \cos\frac{\alpha - \beta}{2}
  \end{equation}
  \begin{equation}
    \cos\alpha - \cos\beta = -2 \sin\frac{\alpha + \beta}{2} \sin\frac{\alpha - \beta}{2}
  \end{equation}

  \subsection*{Formule di Werner}
  \begin{equation}
    \sin\alpha \, \sin\beta = \frac{1}{2} [ \cos(\alpha - \beta) - \cos(\alpha + \beta)]
  \end{equation}
  \begin{equation}
    \cos\alpha \, \cos\beta = \frac{1}{2} [\cos(\alpha + \beta) + \cos(\alpha - \beta)]
  \end{equation}
  \begin{equation}
    \sin\alpha \, \cos\beta = \frac{1}{2} [\sin(\alpha + \beta) + \sin(\alpha - \beta)]
  \end{equation}
\end{document}
